\documentclass{article}
\usepackage{graphicx}
\usepackage{float}

\begin{document}
\title{Amazon Fargate}
\maketitle
	\begin{abstract}

AWS Fargate is a technology for Amazon ECS and EKS* that allows you to run containers without having to manage servers or clusters. With AWS Fargate, you no longer have to provision, configure, and scale clusters of virtual machines to run containers. This removes the need to choose server types, decide when to scale your clusters, or optimize cluster packing. AWS Fargate removes the need for you to interact with or think about servers or clusters. Fargate lets you focus on designing and building your applications instead of managing the infrastructure that runs them.\\

Amazon ECS and EKS have two modes: Fargate launch type and EC2 launch type. With Fargate launch type, all you have to do is package your application in containers, specify the CPU and memory requirements, define networking and IAM policies, and launch the application. EC2 launch type allows you to have server-level, more granular control over the infrastructure that runs your container applications. With EC2 launch type, you can use Amazon ECS and EKS to manage a cluster of servers and schedule placement of containers on the servers. Amazon ECS and EKS keeps track of all the CPU, memory and other resources in your cluster, and also finds the best server for a container to run on based on your specified resource requirements. You are responsible for provisioning, patching, and scaling clusters of servers. You can decide which type of server to use, which applications and how many containers to run in a cluster to optimize utilization, and when you should add or remove servers from a cluster. EC2 launch type gives you more control of your server clusters and provides a broader range of customization options, which might be required to support some specific applications or possible compliance and government requirements. \cite{ref:fargate}


	\bibliographystyle{IEEEtran}
	\bibliography{/Users/viaandad/cloudmesh/hid-sp18-522/technology/references}
\end{abstract}
	
\end{document}